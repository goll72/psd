\documentclass[a4paper,12pt]{report}

\usepackage{float}

\usepackage[brazilian]{babel}
\usepackage[T1]{fontenc}
\usepackage[indent=20pt]{parskip}

\usepackage{tikz}
\usetikzlibrary{calc,automata,positioning,arrows,shapes}

\usepackage[siunitx]{circuitikz}

\usepackage[margin=2cm,bottom=3cm]{geometry}

\ctikzset{logic ports=ieee}

\title{Relatório --- CPU}
\date{\today}

\begin{document}

\maketitle

\section*{Introdução}

A \textit{Central Processing Unit} (\textit{CPU}) é um componente essencial
na arquitetura atual de computadores. Na forma mais comumente encontrada atualmente,
consiste de um circuito sequencial digital composto por registradores, somadores
e outros circuitos lógicos combinacionais e apresenta uma unidade de controle,
que pode ser concebida como uma máquina de estados (\textit{FSM}), responsável 
por enviar sinais de controle para sincronizar e arbitrar a operação de todos 
os outros componentes, de acordo com as instruções a serem executadas.

Nesse projeto, foi desenvolvida uma \textit{CPU} digital de 8 \textit{bits} (sendo
esse o tamanho de uma palavra, dos registradores e dos barramentos de dados e endereços).
A \textit{CPU} é capaz de realizar diversas operações lógicas e aritméticas 
envolvendo valores armazenados nos registradores, mover valores entre registradores,
carregar valores da memória para os registradores e vice-versa. Além disso, também é 
capaz de realizar saltos condicionais e incondicionais. Visto que a largura do 
barramento de endereço é 8 \textit{bits}, podemos endereçar apenas 256 \textit{bytes} 
de memória. 

Valores inteiros são representados em complemento de dois, permitindo
representar valores com sinal entre -128 e 127 e valores sem sinal entre 0 e 255.
Essa representação é extremamente simples e eficaz, uma vez que o resultado de
uma operação de soma ou subtração é o mesmo independentemente dos operandos serem 
tratados como valores com sinal ou sem sinal.

\section*{Instruções}

A \textit{CPU} implementada apresenta instruções de tamanho variável, de 8 ou 16 
\textit{bits}. O primeiro \textit{byte} é fixo, apresentando o seguinte formato:

\[
	oooo\ aa\ bb, \quad \textrm{\textit{o}: opcode}, \; \textrm{\textit{a}: registrador}, \; \textrm{\textit{b}: registrador}
\]

Caso $a$ ou $b$ se refiram ao registrador imediato, o próximo \textit{byte} na 
memória será lido e guardado nesse registrador antes de executar a instrução. Note que,
nos itens a seguir, $a$ e $b$ correspondem aos registradores referidos na instrução, 
não ao registrador A ou B.

\begin{itemize}
	\item ADD: soma os valores guardados nos registradores informados na instrução, armazena o resultado no registrador R
	\item SUB: subtrai o valor guardado no registrador $a$ com o valor guardado do registrador $b$, armazena o resultado em R
	\item OR: realiza a operação ``OU bit-a-bit'' do valor guardados no registrador $a$ com o valor guardado do registrador $b$, armazena o resultado em R
	\item NOT: inverte os bits do valor guardado no registrador $a$, amazena o resultado em R
	\item AND: realiza operação ``E bit-a-bit'' entre o valor guardado no registrador $a$ com o valor guardado do registrador $b$, armazena o resultado em R
	\item CMP: compara o valor do registrador $a$ com o valor do registrador $b$, através da operação SUB, porém não armazena em R
	\item JMP: salta para o endereço informado pelo registrador $b$
	\item JEQ: salta para o endereço informado pelo registrador $b$, se a \textit{flag} Z for igual a 1 
	\item JGR: salta para o endereço informado pelo registrador $b$, se a \textit{flag} S for igual a 1.

	\textit{Obs.\ JLT seria um nome mais coerente para essa instrução.}

	\item LOAD: carrega um valor da memória para o registrador $a$ no endereço dado pelo registrador $b$
	\item STORE: armazena o valor do registrador $a$ no endereço de memória dado pelo registrador $b$
	\item MOV: sobrescreve o registrador $a$ com o valor atual o registrador $b$
	\item IN: realiza uma requisição de entrada para a unidade de I/O, lê o valor de entrada e armazena no registrador $a$
	\item OUT: realiza uma requisição de saída para a unidade de I/O, escreve o valor dado pelo registrador $a$
	\item WAIT: espera até que o sinal {\footnotesize INT} seja habilitado.

	Para a implementação da instrução WAIT na placa \textit{FPGA}, foi usada uma \textit{FSM} para detectar
	sequências de ``uns'' no sinal {\footnotesize INT} e transformá-las em um pulso com duração de um único ciclo
	de \textit{clock}, a fim de permitir que um usuário pressione o botão correspondente e tenha a garantia de que 
	apenas uma instrução WAIT será processada durante esse período.

	\item NOP: nenhuma operação é realizada	
\end{itemize}

\section*{Componentes externos}

\subsection*{CPU}

Como na figura \ref{fig:comp_ext}, tratamos a \textit{CPU} como um bloco lógico, possuindo dessa
forma sinais de \textit{inputs} e \textit{output} de dados.  Temos dois barramantos de oito 
bits, o barramento mais acima permite tanto a entrada quanto a saída de dados com a Unidade de I/O, 
além de permitir a comunicação de dados com a memória (o bloco \textit{MEM}). O último
barramento tem a função de passar o endereço de memória onde será lido ou escrito tal dado.

\subsection*{Unidade de I/O}

Tal bloco possui a função de comunicação entre o usuário com a \textit{CPU}, logo, possui
apenas um único barramento de oito bits, funcionado como \textit{input} do usuário para a CPU
quanto \textit{output} da \textit{CPU} para o usuário. 

Em nossa implementação a Unidade de I/O possui um registrador, que guarda o último valor 
recebido de \textit{output} da \textit{CPU}. Enquanto não é enviado um novo valor e o sinal que 
habilita a saída desse dado, tal valor guardado continuará sendo mostrado para o usuário. Outro
detalhe de implementação da unidade de I/O é outro registrador de um bit, que guarda 1 se a saída
realmente for válida, enquanto o mesmo for zero, indica que até o momento não teve nenhuma saída do sistema,
 se tornando 1, após a primeira saída.

\subsection*{Memória}

O Bloco de memória pode ser compreendido como uma matriz de 256 linhas por 8 colunas, na qual o
endereço passado pelo barramento com a CPU seleciona uma dessas linhas, e sua saída é o valor
guardado nas colunas dessa linha, que será disponivel no barramento superior com a \textit{CPU}
na próxima borda de subida do \textit{clock}.

\begin{figure}[H]
\centering
\begin{circuitikz}[
	>=Triangle,
	square/.style={regular polygon,regular polygon sides=4},
	abstractic/.style={square,thick,draw,minimum size=2.5cm}
]

	\draw (3,4) node[abstractic] (io) {I/O};
	\draw (0,0) node[abstractic] (cpu) {CPU};
	\draw (6,0) node[abstractic] (mem) {MEM};

	\draw[<->,thick] ([yshift=1em]cpu.east) -- ([yshift=1em]mem.west);
	\draw[->,thick] ([yshift=-1em]cpu.east) -- ([yshift=-1em]mem.west);

	\path (cpu) -- (mem) node[midway] (mid) {};

	\draw[->,thick] ([yshift=1em]mid.center) to[short,*-] (io.south);

	\draw[scale=0.8,thick,transform shape] ([yshift=1.25em]cpu.east) ++ (0.2,0) to[multiwire=8] ([yshift=1.25em]mid.center);
\end{circuitikz}

\caption{\label{fig:comp_ext}Componentes externos}
\end{figure}

\newpage

\section*{FSM}

A unidade de controle foi implementada usando uma \textit{FSM} do tipo \textit{Mealy}, cujo diagrama na forma de grafo
direcionado é exibido na figura \ref{fig:fsm}. Para simplificar o diagrama, as arestas não indicam as condições nem as 
saídas para cada transição, além de colapsar várias transições entre os mesmos estados em uma única aresta.

Enquanto o sinal de \textit{reset} do circuito estiver acionado, estamos no primeiro estado da \textit{FSM}, {\footnotesize RESET}.
Em seguida, o estado {\footnotesize FETCH} é atingido, nesse momento é buscado o conteúdo no endereço de memória que se
encontra no PC (\textit{Progam Counter}). Depois temos o estado {\footnotesize STORE}, em que o valor encontrado
é guardado nos registradores IR (\textit{Instruction Register}) e RS (\textit{Register Select}),
caso na instrução esteja indicado que é necessário o imediato, buscamos ele na memória durante o estado
{\footnotesize FETCH IMM} e depois é colocado no registrador durante o estado {\footnotesize STORE IMM}.
Por fim executamos tal instrução. Caso a instrução não seja uma operação WAIT, volta-se para o estado
{\footnotesize FETCH} novamente. senão entra-se no estado {\footnotesize POLL}, na qual é aguardado até o sinal
{\footnotesize INT} ser acionado, voltando para o estado {\footnotesize FETCH}.

\begin{figure}[H]
\advance\leftskip+1.8cm

\begin{tikzpicture}[scale=0.8, every node/.style={transform shape}]
	\node[state, minimum size=2cm, initial] (reset) {RESET};
	\node[state, minimum size=2cm, right=of reset] (fetch) {FETCH};
	\node[state, minimum size=2cm, right=of fetch] (store) {STORE};

	\node[state, minimum size=2cm, above right=of store] (execute) {\footnotesize EXECUTE};
	\node[state, minimum size=2cm, right=of execute] (poll) {POLL};

	\node[state, minimum size=2cm, below right=of store] (fetch_imm) {\small \shortstack{FETCH\\IMM}};
	\node[state, minimum size=2cm, right=of fetch_imm] (store_imm) {\small \shortstack{STORE\\IMM}};

	\draw[->,thick] (reset) edge (fetch)
	          (fetch) edge (store)
	          (store) edge[bend left] (execute)
	          (store) edge[bend right] (fetch_imm)
	          (fetch_imm) edge (store_imm)
	          (store_imm) edge[bend right] (execute)
	          (execute) edge (poll)
	          (poll) edge[out=120,in=60,looseness=3] (poll)
	          (execute) edge[out=90,in=90,out looseness=1,in looseness=2] (fetch)
	          (poll) edge[out=315,in=270,distance=10cm] (fetch.south);
\end{tikzpicture}

\vspace{-92pt}
\caption{\label{fig:fsm}Diagrama de estados}
\end{figure}

\section*{Componentes internos}

\begin{figure}[H]
\centering
\begin{circuitikz}[
	>=Triangle,
	scale=0.6,
	transform shape
]
	\tikzset{conn/.style={
		short,
		nodes width=0.07
	}}

	\tikzset{ic/.style={
		muxdemux,
		circuitikz/muxdemux/inner label font={},
		circuitikz/muxdemux/outer label font={},
		circuitikz/muxdemux/clock wedge size=0.5,
		circuitikz/muxdemux/inner label xsep=8pt,
		circuitikz/muxdemux/inner label ysep=6pt
	}}

	\tikzset{control_unit/.style={
		ic,
		muxdemux def={Lh=6.5,Rh=6.5,w=6.5,NL=5,NT=2,NB=0,NR=5},
		draw only left pins={2,4,5},
		draw only right pins={3,5},
		muxdemux label={L2=INT,L4=IR,L5=RS,T1=S,T2=Z,R3=CTL,R5=SEL}
	}}

	\tikzset{reg_file/.style={
		ic,
		muxdemux def={Lh=7.5,Rh=7.5,w=7,NL=5,NT=0,NB=0,NR=2},
		draw only left pins={1,2,4,5},
		muxdemux label={L1=D,L2=SEL,L4=$\textrm{SEL}_A$,L5=$\textrm{SEL}_B$,R1=A,R2=B},
	}}

	\tikzset{alu/.style={
		ic,
		muxdemux def={Lh=6.5,Rh=3,NL=2,NR=1,NT=1,NB=2,w=3,inset w=0.75,inset Lh=1,inset Rh=0},
		muxdemux label={B1=O,B2=C}
	}}

	\tikzset{reg/.style={
		flipflop,
		flipflop def={t1=D,t6=Q}
	}}

	\tikzset{reg_wren/.style={
		reg,
		flipflop def={t1=D,t3=$\textrm{WR}_{EN}$,t6=Q}
	}}

	\coordinate (left_offset_near) at ($(-1,0)$);
	\coordinate (left_offset_far) at ($(-1.5,0)$);

	% signals
	\draw (2,18) node[signal,draw,signal to=east,anchor=east] (int) {\footnotesize INT};
	\draw (2,17) node[signal,draw,signal to=east,anchor=east] (data_bus) {\footnotesize DATA\_BUS};
	\draw (21,-6) node[signal,draw,signal to=west] (addr_bus) {\footnotesize ADDR\_BUS};

	% control unit and register file
	\draw (4, 14) node[adder] (pc_adder) {};
	\draw (10,10) node[control_unit,label={[label distance=8pt]south:CONTROL UNIT}] (cu) {};
	\draw (8,0) node[reg_file,label={[label distance=8pt]south:REGISTER FILE}] (regs) {};

	\coordinate (alu_xpos) at ($(regs.center) + (5,0)$);
	\draw (regs.rpin 1 -| alu_xpos) node[alu,anchor=lpin 1] (alu) {};

	% register connections
	\draw (regs.rpin 1) edge (alu.lpin 1)
	      (regs.rpin 2) edge (alu.lpin 2);

	% registers
	\coordinate (pc_ir_rs_start) at (0,10);
	\coordinate (reg_offset) at (0,-4);

	\draw (pc_ir_rs_start) node[reg,label={[label distance=6pt]south:PC}] (pc) {};
	\draw ($(pc_ir_rs_start) + 1*(reg_offset)$) node[reg_wren,label={[label distance=6pt]south:IR}] (ir) {};
	\draw ($(pc_ir_rs_start) + 2*(reg_offset)$) node[reg_wren,label={[label distance=6pt]south:RS}] (rs) {};
	
	\coordinate (zsco_start) at (21,14);

	\draw (zsco_start) node[reg_wren,label={[label distance=6pt]south:Z}] (z) {};
	\draw ($(zsco_start) + 1*(reg_offset)$) node[reg_wren,label={[label distance=6pt]south:S}] (s) {};
	\draw ($(zsco_start) + 2*(reg_offset)$) node[reg_wren,label={[label distance=6pt]south:C}] (c) {};
	\draw ($(zsco_start) + 3*(reg_offset)$) node[reg_wren,label={[label distance=6pt]south:O}] (o) {};

	% pc adder connections
	\coordinate (pc_adder_out_xpos) at ($(pc.center) + (-2.5,0)$);
	\draw (pc.pin 6) -- (pc.pin 6 -| pc_adder.south);
	\draw[->] (pc_adder.south |- pc.pin 6) -- (pc_adder.south); 

	\node at ([yshift=4em]pc_adder.north) (pc_adder_const_one) {1};
	\draw[->] ([yshift=3em]pc_adder.north) -- (pc_adder.north);
	\draw (pc_adder.west) -- (pc_adder.west -| pc_adder_out_xpos);

	\draw ($(pc_adder.west -| pc_adder_out_xpos)!0.5!(pc_adder_out_xpos |- pc.pin 1)$) 
	      node[buffer port,scale=0.8,anchor=center,rotate=270] (next_pc_tristate) {};

	\draw (pc_adder.west -| pc_adder_out_xpos) -- (next_pc_tristate.in 1);
	\draw (next_pc_tristate.out) -- (pc_adder_out_xpos |- pc.pin 1);
	\draw (pc_adder_out_xpos |- pc.pin 1) to[conn,*-] (pc.pin 1);

	\draw (next_pc_tristate.up) -- ($(next_pc_tristate.up) + (0.3,0)$) 
	      node[signal,draw,signal to=west,anchor=west] {\footnotesize CTL\_INCREMENT\_PC};

	% int signal to control unit
	\coordinate (cu_int_offset) at ($(cu.lpin 2) + (left_offset_near)$);
	\draw (int.east) -- (int.east -| cu_int_offset) -- (cu_int_offset |- cu.lpin 2) -- (cu.lpin 2);

	% ir to control unit
	\draw (ir.pin 6) -- (ir.pin 6 -| alu.tpin 1) -- (alu.tpin 1);
	\coordinate (cu_ir_offset) at ($(cu.lpin 4) + (left_offset_near)$);
	\draw (cu.lpin 4) -- (cu_ir_offset) to[conn,-*] (cu_ir_offset |- ir.pin 6);

	% pc output connections
	\draw (pc.pin 6 -| pc_adder.south) to[conn,*-] (pc_adder.south |- addr_bus);

	\draw ($(pc_adder.south |- addr_bus)!0.2!(addr_bus.west)$) 
	      node[buffer port,scale=0.8,anchor=center] (pc_addr_tristate) {};

	\draw (pc_adder.south |- addr_bus) -- (pc_addr_tristate.in 1);
	\draw (pc_addr_tristate.out) -- (addr_bus.west);

	\coordinate (pc_addr_signal_pos) at ($(pc_addr_tristate.center) + (-0.5,-0.9)$);

	\draw (pc_addr_tristate.down) -- (pc_addr_tristate.down |- pc_addr_signal_pos) -- (pc_addr_signal_pos) 
	      node[signal,draw,signal to=east,anchor=east] {\footnotesize CTL\_PC\_TO\_ADDR};

	% register file b output to addr
	\coordinate (reg_b_addr_out) at ($(regs.rpin 2)!0.75!(alu.lpin 2)$);

	\draw ($(reg_b_addr_out)!0.8!(reg_b_addr_out |- addr_bus)$) node[buffer port,scale=0.8,rotate=270] (reg_b_addr_tristate) {};

	\draw (reg_b_addr_out) to[conn,*-] (reg_b_addr_out |- reg_b_addr_tristate.in 1);
	\draw (reg_b_addr_tristate.out) to[conn,-*] (reg_b_addr_out |- addr_bus);

	\draw (reg_b_addr_tristate.up) -- ($(reg_b_addr_tristate.up) + (0.75,0)$);
	\draw ($(reg_b_addr_tristate.up) + (0.75,0)$) node[signal,draw,signal to=west,anchor=west] {\footnotesize CTL\_REG\_B\_TO\_ADDR};

	% register file b output to pc
	\coordinate (reg_b_pc_out) at ($(reg_b_addr_out)!0.85!(reg_b_addr_out |- reg_b_addr_tristate.in 1)$);

	\draw ($(reg_b_pc_out)!0.8!(pc_adder_out_xpos |- reg_b_pc_out)$)
	      node[buffer port,scale=0.8,rotate=180] (reg_b_pc_tristate) {};

	\coordinate (reg_b_pc_signal_pos) at ($(reg_b_pc_tristate) + (-0.5,-0.85)$);

	\draw (reg_b_pc_tristate.up) -- (reg_b_pc_tristate.up |- reg_b_pc_signal_pos) -- (reg_b_pc_signal_pos)
	      node[signal,draw,signal to=east,anchor=east] {\footnotesize CTL\_REG\_B\_TO\_PC};

	\draw (reg_b_pc_out) to[conn,*-] (reg_b_pc_tristate.in 1);
	\draw (reg_b_pc_tristate.out) -- (reg_b_pc_tristate.out -| pc_adder_out_xpos) -- (pc_adder_out_xpos |- pc.pin 1);

	% data bus to register file data in
	\coordinate (data_bus_reg_wire_xpos) at ($(pc_adder) + (1,0)$);
	\coordinate (data_bus_reg_tristate_pos) at ($(data_bus_reg_wire_xpos |- ir.pin 6)!0.5!(regs.lpin 1 -| data_bus_reg_wire_xpos)$);
	
	\draw (data_bus_reg_tristate_pos) node[buffer port,scale=0.8,rotate=270] (data_bus_reg_tristate) {};
	\draw (data_bus.east) -- (data_bus.east -| data_bus_reg_wire_xpos) -- (data_bus_reg_tristate.in 1);

	\coordinate (reg_file_data_in_short) at (data_bus_reg_tristate.out |- regs.lpin 1);

	\draw (data_bus_reg_tristate.out) to[conn,-*] (reg_file_data_in_short) -- (regs.lpin 1);

	\draw (data_bus_reg_tristate.up) -- ($(data_bus_reg_tristate.up) + (0.3,0)$) 
	      node[signal,draw,signal to=west,anchor=west] {\footnotesize CTL\_DATA\_TO\_REG};

	% alu coordinates
	\coordinate (alu_reg_out_pos) at ($(alu.rpin 1) + (0.5,0)$);
	\coordinate (alu_reg_out_turn_ypos) at ($(alu_reg_out_pos)!0.65!(alu_reg_out_pos |- data_bus_reg_tristate)$);

	% register file a output coordinates
	\coordinate (reg_a_data_out) at ($(regs.rpin 1)!0.25!(alu.lpin 1)$);
	\coordinate (reg_a_data_out_turn_ypos) at ($(reg_a_data_out)!0.8!(reg_a_data_out |- ir.pin 6)$);

	% alu to sz
	\draw ($(z.pin 1) + (-2,0)$) node[nor port,circuitikz/ieeestd ports/pin length=0] (nor_alu_out_z) {};
	\draw (alu_reg_out_turn_ypos) to[conn,*-] (alu_reg_out_turn_ypos |- nor_alu_out_z.body left) -- (nor_alu_out_z.body left);
	\draw (nor_alu_out_z.bout) -- (z.pin 1);

	\draw (alu_reg_out_pos |- s.pin 1) to[conn,*-] (s.pin 1);

	% register file a output to data bus
	\draw ($(reg_a_data_out)!0.65!(reg_a_data_out |- ir.pin 6)$) node[buffer port,rotate=90,scale=0.8] (reg_a_data_tristate) {};
	\draw (reg_a_data_tristate.out) -- (reg_a_data_out_turn_ypos) to[conn,-*] (reg_a_data_out_turn_ypos -| data_bus_reg_wire_xpos); 

	\draw ($(reg_a_data_tristate) + (1,0)$) node[signal,draw,signal to=west,anchor=west,fill=white] (reg_a_data_signal) {\footnotesize CTL\_REG\_A\_TO\_DATA};
	\draw (reg_a_data_tristate.down) -- (reg_a_data_signal.west);

	\draw (reg_a_data_out) to[conn,*-] (reg_a_data_tristate.in 1);

	% register file b output to register file data in
	\coordinate (reg_b_reg_out_xpos) at ($(reg_b_pc_out)!0.7!(pc_adder_out_xpos |- reg_b_pc_out)$);

	\draw ($(reg_b_reg_out_xpos) + (0,1.5)$) node[buffer port,rotate=90,scale=0.8] (reg_b_reg_tristate) {};
	\draw (reg_b_reg_out_xpos) to[conn,*-] (reg_b_reg_tristate.in 1);
	\draw (reg_b_reg_tristate.out) -- (reg_b_reg_tristate.out |- reg_file_data_in_short) -- (reg_file_data_in_short);

	\draw ($(reg_b_reg_tristate) + (-1,0)$) node[signal,draw,signal to=east,anchor=east,fill=white] (reg_b_reg_signal) {\footnotesize CTL\_REG\_B\_TO\_REG};
	\draw (reg_b_reg_tristate.up) -- (reg_b_reg_signal);

	% alu output to register file data in
	\draw ($(alu_reg_out_turn_ypos)!0.3!(alu_reg_out_turn_ypos -| data_bus_reg_wire_xpos)$) node[buffer port,scale=0.8,rotate=180] (alu_reg_tristate) {};
	
	\draw (alu.rpin 1) -- (alu_reg_out_pos) -- (alu_reg_out_turn_ypos) -- (alu_reg_tristate.in 1);
	\draw (alu_reg_tristate.out) to[conn,-*] (alu_reg_out_turn_ypos -| data_bus_reg_wire_xpos);

	\draw ($(alu_reg_tristate) + (0.5,1)$) node[signal,draw,signal to=west,anchor=west,fill=white] (alu_reg_signal) {\footnotesize CTL\_ALU\_TO\_REG};

	\draw (alu_reg_tristate.down) -- (alu_reg_tristate.down |- alu_reg_signal) -- (alu_reg_signal.west);

	% zsco enable inputs
	\draw ($(z.pin 3) + (-0.5,0)$) node[signal,draw,signal to=east,anchor=east,fill=white] {\footnotesize CTL\_ALU\_TO\_REG};
	\draw ($(z.pin 3) + (-0.5,0)$) -- (z.pin 3);

	\draw ($(s.pin 3) + (left_offset_near)$) -- (s.pin 3)
	  ($(c.pin 3) + (left_offset_near)$) -- (c.pin 3)
	  ($(o.pin 3) + (left_offset_near)$) -- (o.pin 3);

	\draw ($(s.pin 3) + (left_offset_near)$)
	                 to[conn, -*] ($(c.pin 3) + (left_offset_near)$)
	                 to[conn, -*] ($(o.pin 3) + (left_offset_near)$);

	\coordinate (update_sco_wire_pos) at ($(o.pin 3) + (left_offset_near)$);
	\coordinate (update_sco_signal_pos) at (update_sco_wire_pos |- reg_b_pc_out);

    \draw (update_sco_wire_pos) -- (update_sco_signal_pos) -- ($(update_sco_signal_pos) + (-0.75,0)$) node[signal,draw,signal to=east,anchor=east] {\footnotesize CTL\_UPDATE\_SCO};

	% alu to co
	\coordinate (alu_c_ypos) at ($(alu.bpin 2) + (0,-0.5)$);
	\coordinate (alu_c_xpos) at ($(update_sco_wire_pos) + (-1,0)$);
	\coordinate (alu_c_origin) at (alu_c_ypos -| alu_c_xpos);
	
	\draw (alu.bpin 2) -- (alu_c_ypos) -- (alu_c_origin) -- (alu_c_origin |- c.pin 1) -- (c.pin 1);

	\coordinate (alu_o_ypos) at ($(alu.bpin 1) + (0,-0.5)$);
	\coordinate (alu_o_xpos) at ($(update_sco_wire_pos) + (-0.5,0)$);
	\coordinate (alu_o_origin) at (alu_o_ypos -| alu_o_xpos);
	
	\draw (alu.bpin 1) -- (alu_o_ypos) -- (alu_o_origin) -- (alu_o_origin |- o.pin 1) -- (o.pin 1);
	
	% zs to control unit
	\coordinate (z_turn_ypos) at ($(z.pin 6) + (0.5,1)$);
	\coordinate (s_turn_ypos) at ($(z_turn_ypos) + (0,0.5)$);
	
	\coordinate (z_xpos) at ($(z.pin 6) + (0.5,0)$);
	\coordinate (s_xpos) at ($(s.pin 6) + (1,0)$);
	
	\draw (z.pin 6) -- (z_xpos) -- (z_xpos |- z_turn_ypos) -- (z_turn_ypos -| cu.tpin 2) -- (cu.tpin 2);
	\draw (s.pin 6) -- (s_xpos) -- (s_xpos |- s_turn_ypos) -- (s_turn_ypos -| cu.tpin 1) -- (cu.tpin 1);

	% carry, overflow register outputs
	\draw (c.pin 6) -- ($(c.pin 6) + (0.5,0)$) node[plain crossing,scale=1.2,rotate=45] {};
	\draw (o.pin 6) -- ($(o.pin 6) + (0.5,0)$) node[plain crossing,scale=1.2,rotate=45] {};

	% "control" signal
	\draw (cu.rpin 3) -- ($(cu.rpin 3) + (0.5,0)$) node[signal,draw,signal to=west,anchor=west] {\footnotesize CTL};
	\draw (cu.rpin 5) -- ($(cu.rpin 5) + (0.5,0)$) node[signal,draw,signal to=west,anchor=west] {\footnotesize SEL};

	% rs connections
	\coordinate (rs_short) at ([xshift=4.5em]rs.pin 6);
	\draw (rs.pin 6) to[conn,-*] (rs_short);

	\draw (rs_short) -- (rs_short |- cu.lpin 5) -- (cu.lpin 5);
	\draw (rs_short) -- (rs_short |- regs.lpin 4) to[conn,-*] (regs.lpin 4 -| data_bus_reg_wire_xpos);

	\draw (regs.lpin 4 -| data_bus_reg_wire_xpos) -- (regs.lpin 4);
	\draw (regs.lpin 4 -| data_bus_reg_wire_xpos) -- (data_bus_reg_wire_xpos |- regs.lpin 5) -- (regs.lpin 5);

	% data bus to ir
	\coordinate (data_bus_ir_short) at ($(data_bus.east) + (0.75,0)$);
	\coordinate (data_bus_ir_turn) at ($(pc_adder) + (0,1)$);
	\coordinate (data_bus_ir_2nd_turn) at ($(pc_adder_out_xpos) + (-1,0)$);
	
	\draw (data_bus_ir_short) to[conn,*-] (data_bus_ir_short |- data_bus_ir_turn) -- (data_bus_ir_turn -| data_bus_ir_2nd_turn) to[conn,-*] (data_bus_ir_2nd_turn |- ir.pin 1) -- (data_bus_ir_2nd_turn |- rs.pin 1);
	\draw (data_bus_ir_2nd_turn |- ir.pin 1) -- (ir.pin 1);
	\draw (data_bus_ir_2nd_turn |- rs.pin 1) -- (rs.pin 1);

	\draw ($(ir.pin 3) + (-0.5,0)$) node[signal,draw,signal to=east,anchor=east,fill=white] {\footnotesize CTL\_DATA\_TO\_IR} -- (ir.pin 3);
	\draw ($(rs.pin 3) + (-0.5,0)$) node[signal,draw,signal to=east,anchor=east,fill=white] {\footnotesize CTL\_DATA\_TO\_IR} -- (rs.pin 3);

	% register file "sel" signal
	\draw ($(regs.lpin 2) + (-0.35,0)$) node[signal,draw,signal to=east,anchor=east] {\footnotesize SEL} -- (regs.lpin 2);
\end{circuitikz}

\caption{\label{fig:circuit} Esquemática do circuito da \textit{CPU}}
\end{figure}

Na figura \ref{fig:circuit}, é exibido um diagrama do circuito interno da \textit{CPU}.
Para simplificar o diagrama, barramentos não são marcados como tal e o sinal
$\textrm{WR}_{EN}$ do PC e do banco de registradores não é exibido (esse sinal 
corresponde ao OU lógico de todos os sinais de controle que podem habilitar 
caminhos que chegam na entrada D desses registradores). Os sinais 
CTL, IR, RS, $\textrm{SEL}_A$ e $\textrm{SEL}_B$ são 
barramentos de 15, 4, 4, 2 e 2 \textit{bits}, respectivamente. As entradas e saídas da
Unidade Lógica e Aritmética (ULA) são barramentos de 8 \textit{bits}. Os registradores
Z, S, C e O são registradores de 1 \textit{bit}.

\subsection*{Unidade de Controle}

A unidade de controle tem como entrada o sinal {\footnotesize INT}, a instrução atual e as 
\textit{flags} S e Z. As saídas são os sinais de controle e os sinais
de seleção do registrador a ser sobrescrito no banco de registradores. Como já mencionado,
foi implementada usando uma máquina de estados do tipo \textit{Mealy}.

\subsubsection*{Sinais de controle}

Os sinais de controle podem ser divididos em três tipos: os que agem como \textit{gate}
para um outro sinal, como é o caso do sinal {\footnotesize CTL\_UPDATE\_SCO}, que permitirá
que os registradores S, C e O tenham seu conteúdo atualizado de
acordo com as entradas atuais; os que são saídas da \textit{CPU} e, portanto, não são 
exibidos no diagrama, como o {\footnotesize CTL\_MEM\_EN}, que habilita a memória, e os que 
servem para habilitar ou desabilitar \textit{buffers tri-state}, que são uma forma simples
e eficiente de ``juntar'' vários sinais e escolher qual sinal poderá passar em um 
determinado momento (age como um \textit{switch} e é, de fato, comumente implementado 
usando transistores).

É interessante notar que, ao realizar a síntese do circuito para uma placa \textit{FPGA},
todos os \textit{buffers tri-state} utilizados são convertidos em multiplexadores, uma vez 
que a \textit{FPGA} é incapaz de representar esse tipo de tecnologia. Há uma exceção, no
entanto, que é o \textit{buffer tri-state} habilitado pelo sinal 
{\footnotesize CTL\_REG\_A\_TO\_DATA}, uma vez que se trata de saída em um barramento
compartilhado por vários dispositivos (em particular, a unidade de memória).

\subsubsection*{Clock}

É importante notar que, na implementação da \textit{CPU}, o \textit{clock} da unidade de
controle e do banco de registradores está invertido com relação ao \textit{clock} do resto
do sistema (memória, I/O, PC, etc.). Ou ainda, equivalentemente, poderíamos dizer que esses
componentes são disparados por borda de descida usando o mesmo sinal de \textit{clock}.

Isso permite que um valor que foi buscado na memória seja obtido e possa ser utilizado 
imediatamente, na próxima borda de subida. Caso tanto a unidade de controle quanto a memória
fossem disparados pela mesma borda do mesmo sinal de \textit{clock}, ao buscar um valor 
da memória, esse valor se tornaria disponível apenas na próxima borda do \textit{clock} e 
só poderia ser utilizado (armazenado em um registrador) na borda de \textit{clock} subsequente, 
desperdiçando um ciclo de \textit{clock}.

\subsection*{Registradores de \textit{flags}}

Há quatro registradores de \textit{flags}: Z, S, C e O,
embora apenas dois (Z e S) sejam utilizados.

O registrador Z guarda 1 caso o último resultado da ULA tenha sido zero (o que corresponde
a igualdade quando a operação subtração ou comparação é realizada). Já o registrador S guarda o 
\textit{bit} mais significativo do resultado da última operação aritmética (soma, subtração ou 
comparação), que corresponde ao sinal quando o resultado é interpretado como um número com
sinal.

Os registradores C e O guardam, respectivamente, as condições de \textit{carry out}
e \textit{overflow} geradas pelas operações de soma, subtração e comparação. A condição de 
\textit{carry out} indica que o resultado da operação não pôde ser representado em 8 \textit{bits} 
quando tratado como um valor sem sinal. Já a condição de \textit{overflow} indica que o resultado 
da operação não pôde ser representado em 8 \textit{bits} quando tratado como um valor com sinal.

\subsection*{Banco de registradores}

O banco de registradores coordena o acesso aos registradores de uso geral da \textit{CPU},
que são 4: A, B, R e I. O registrador R é o registrador acumulador, ou seja, esse registrador
é implicitamente usado como registrador de destino em todas as instruções que produzem um valor 
que deve ser armazenado em algum registrador.

Além disso, o registrador I também é tratado especialmente, uma vez que é usado 
para armazenar imediatos em instruções que especifiquem um operando imediato. Se trata 
de um detalhe de implementação e, portanto, não pode ser acessado diretamente ao nível 
do \textit{assembly}, embora possa ser, em tese, manipulado como qualquer outro registrador. 
Sua inclusão simplifica o tratamento de endereços de registradores, uma vez que há 4 endereços 
possíveis (mesmo se apenas 3 registradores fossem utilizados).

O banco de registradores consiste de um decodificador n-para-$2^n$, que fornece o sinal 
$\textrm{WR}_{EN}$ de cada registrador, garantindo que apenas um registrador poderá ser 
sobrescrito para cada ciclo de \textit{clock}. Além disso, decodificadores também são usados
para escolher quais registradores serão lidos a cada ciclo de \textit{clock}.

O \textit{design} escolhido para o banco de registradores, inspirado no banco de registradores
da arquitetura \textit{MIPS}, permite que dois registradores sejam lidos a cada ciclo de \textit{clock}.
Dessa forma, as duas saídas do banco de registradores podem ser sempre ligadas à ULA para serem 
processadas diretamente. Além disso, esse \textit{design} permite que apenas uma das saídas seja 
reservada para se conectar ao barramento de dados e a outra seja reservada para se conectar ao 
barramento de endereço.

\subsection*{Unidade Lógica e Aritmética}

A Unidade Lógica e Aritmética (ULA) é composta por um conjunto de circuitos combinacionais que
implementam cada operação lógica e aritmética e, por fim, um multiplexador, para escolher a saída
que deverá ser a saída da ULA. Em particular para a soma e subtração, o mesmo circuito somador 
completo de 8 \textit{bits} é usado, apenas realizando o complemento de dois do segundo valor de
entrada caso a operação seja subtração. 

Vale mencionar, ainda, que a operação comparação foi implementada de forma idêntica à subtração. 
A única diferença é a existência de uma lógica adicional na unidade de controle para evitar que 
o registrador R seja sobrescrito quando essa instrução for executada.

\end{document}
